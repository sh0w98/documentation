\documentclass{letter}
\signature{Gabriel Faucher, SDW20}
\address{Tammingastraat 28\\ Hornhuizen \\ 9978 PC}
\longindentation=0pt
\begin{document}

\begin{letter}{Examencommissie \\ AVO-vakken \\ MBO}
\opening{Hoogeachte Examencomissie,}

Ik schrijf u naar aanleiding van een verzoek van mijn mentor en SOB-er, ten einde mijn
verzoek vervroegd examens af te leggen aan u toe te lichten.
Allereerst zou ik u graag wat achtergrond geven over wie ik ben.
Ik ben 21 jaar oud, hou ontzettend erg van leren, lezen en zoveel mogelijk informatie
tot me nemen. Ik lees al veel sinds ik vrij klein ben.
Ik ben opgegroeid op het platteland, waar ik zonder al teveel problemen op academisch front
de basischool afrondde. Daarna kreeg ik de kans om een toelatingsexamen te doen
voor het Preadinius Gymnasium, aangezien ik de basisschool had afgerond in Friesland. 
Hier slaagde ik voor, en ik kon mijn middelbare schoolcarriere beginnen op het Praedinius
Gymnasium te Groningen. Wat daarop volgde waren een aantal jaren die anders liepen dan ik
had verwacht.

Toen ik ik de 3\textsuperscript{e} klas zat overleed een klasgenoot en goede vriend van mij aan
een epileptische aanval. Ik wist niet hoe ik hiermee om moest gaan, dus greep ik naar drugs om
de immense pijn te verdoven. Ik raakte verslaafd aan verschillende drugs, en 
begon die ook dagelijks te gebruiken. Dit had een aantal zaken tot gevolg. Ten eerste begon ik 
ongeloofelijk veel te spijbelen, en was ik nagenoeg nooit op school. Ik gebruikte liever drugs 
dan dat ik in de les zat. Ten tweede kwam ik terecht in een bijzonder ongezonde relatie. 
Deze relatie heeft heel veel invloed op mij gehad,
en heeft mij ook erg getraumatiseerd. Halverwege de 5\textsuperscript{e} klas ging het zo slecht
met mij dat mijn moeder me thuis hield en me zei dat ik moest gaan uitzoeken wat ik wou, omdat het
zo niet langer kon. Ik nam haar advies ter harte, en ontdekte vrij snel dat programmeren mijn grote
liefde was. Hierna volgde een aantal jaren waarin ik probeerde mijzelf te leren programmeren, en 
verschillende opleidingen probeerde te volgen zonder er een af te maken. Dit ging door tot ik ongeveer
een jaar geleden weer terugkwam bij mijn ouders na een gefaalde poging om op kamers te gaan. 
Ik besloot dat ik zo niet langer door wou gaan. Ik meldde me aan voor een interne opname in een 
verslavingskliniek. Ik ging daar 7 maanden geleden heen, en het bleek de beste beslissing te zijn 
die ik ooit heb genomen.

In de kliniek begon mijn 'herstel'. Ik ben eerlijk geworden tegenover mijzelf en anderen, ben mijn 
verleden aangegaan in plaats van er voor weg te rennen, en heb verantwoordelijkheid genomen voor 
wat ik in het verleden heb gedaan. Na mijn terugkomst uit de kliniek ben ik enige tijd  gaan werken 
als vakantiekracht in de landbouwsector, maar dat werk was voor mij fysiek te zwaar.
Daarna ging ik op zoek naar een opleiding die ik nog kon doen op korte termijn, aangezien het schooljaar
al bijna was begonnen. Zodoende meldde ik me aan voor de MBO opleiding Software Development: Web \& Apps 
bij de afdeling Kunst \& Multimedia van het Noorderpoort in Groningen en een paar dagen later werd ik gebeld 
door mijn huidige mentor. Ik ben vanaf het begin in gesprek gegaan met mijn hem en  mijn studentbegeleider 
om ervoor te zorgen dat ik de opleiding om een goede manier zou doorlopen. In deze gesprekken en in de lessen kwam 
al snel naar voren dat ik een aanzienlijke hoeveelheid voorkennis had. Door de paar jaar die ik had besteed aan
de poging mijzelf te leren programmeren, had ik heel veel van de stof die behandeld werd al op een eerder moment 
geleerd. Dit gold ook voor de AVO-vakken. Op het Praedinius heb ik heel veel stof die we nu in de les behandelen
al geleerd. Bijvoorbeel het schrijven van betogen en beschouwingen bij Nederlands, het maken van lees- en luistertoetsen
bij Engels en het discussieren en analyseren van de politiek en de samenleving als geheel bij Maatschappijleer.
Er zijn daarmee ook veel dingen die ik behandeld heb op het Praedinius die al verder gaan dan wat het examen
van mij verwacht. Ik ben gewend om boeken te moeten lezen en een leeslijst te moeten behandelen voor Nederlands
en Engels, en om historische analyses te maken op basis van verschillende bronnen. Vanaf het moment dat ik 
hier in de praktijk achter kwam ben ik dagelijks gesprekken aangegeaan met verschillende docenten, om zo aan 
te geven wat ik al wist, wat ik nog niet wist en wat ik het liefst zou doen tijdens deze lessen. 
Dit alles heeft als gevolg dat ik inmiddels alle stof voor dit semester heb behandeld wat betreft alle 
vakken die met programmeren te maken hebben, en dat ik lees- luistertoetsen voor Engels heb afgerond met twee 
negens. 

Ik heb in mijn 7 jaar op het Preadinius Gymnasium veel kennis opgedaan, naast de kennis die ik mijzelf
heb geleerd of die ik dankzij mijn opvoeding heb verkgregen. Dit heeft ertoe geleid dat ik ervan 
overtuigd ben dat ik de kennis die van mij verwacht word dat ik die heb voor de examens, al heb.
Dat is dan ook de voornaamste reden waarom ik de examens vervroeg zou willen afleggen; bijna alle
kennis die van mij verwacht wordt te hebben voor de examens, heb ik al in mijn bezit. Dit maakt 
de lessen voor mij overbodig, en die tijd besteed ik liever aan andere onderwerken; bijvoorbeeld
de andere vakken of mijn herstel. Ik ben me ook uiterst bewust van mijn valkuilen en karakterfouten,
en daarin wil ik graag eerlijk zijn. Zowel tegenover u, als tegenover mijzelf. Mocht ik op een bepaald
moment ondervinden dat het met teveel word en ik teveel hooi op mijn fork heb genomen, zal ik dat direct
aangeven bij mijn mentor en bij mijn studentbegeleider. Ik ben dan ook absoluut bereid om meteen weer
terug te gaan naar het normale tempo van mijn klas. Als het me niet lukt, ben ik daar heel eerlijk in.
Desondanks denk ik dat ik de examens vervroegd met succes zal afronden.


Al met al vraag ik u te overwegen mij toe te laten vervroegd examen af te leggen voor de AVO-vakken. Ik heb veel voorkennis
opgedaan dankzij mijn tijd op het Praedinius Gymnasium en de tijd die ik doorgebracht hem trachtend mijzelf te leren
programmeren. Ik ben ervan overtuigd dat ik die verantwoordelijkheid aankan, maar ik ben mij ook uiters bewust
van mijn beperkingen en daar zal ik ook altijd in eerlijk in zijn. Mocht ik op de een of andere manier merken 
dat het mij allemaal te veel wordt of haal ik onvoldoendes op de examens, dan zal ik daar uiteraard direct 
mijn verantwoordelijkheid voor nemen en mij neerleggen bij het normale tempo van de lessen en examens.
\closing{Hoogachtend,}

\end{letter}

\end{document}
